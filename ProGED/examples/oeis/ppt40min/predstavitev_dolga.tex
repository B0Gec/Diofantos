\documentclass[t,usenames,dvipsnames]{beamer} % [t] pomeni poravnavo na vrh slida

% \usepackage{etex} % vključi ta paket, če ti javi napako, da imaš naloženih preveč paketov.

% standardni paketi
\usepackage[slovene]{babel}
\usepackage[T1]{fontenc}
\usepackage[utf8]{inputenc}
\usepackage{amssymb}
\usepackage{amsmath}
%\usepackage{diagrams}

% % down 4.4.2017
\newcommand{\R}{\mathbb R}
\newcommand{\N}{\mathbb N}
\newcommand{\Z}{\mathbb Z}
\newcommand{\C}{\mathbb C}
\newcommand{\Q}{\mathbb Q}
\definecolor{softyellow}{rgb}{0.98,0.98,0.75}
\setbeamercolor{loweryel}{fg=black,bg=softyellow}
% % up 4.4.2017

% % down 2/3 prosojnica: crno
\usepackage{tikz}
\newcommand{\fillblack}[1]{
	\begin{tikzpicture}[remember picture, overlay]
	\node [shift={(0 cm,0cm)}]  at (current page.south west)
	{%
		\begin{tikzpicture}[remember picture, overlay] at (current page.south west)
		\draw [fill=black] (0, 0) -- (0,#1 \paperheight) --
		(\paperwidth,#1 \paperheight) -- (\paperwidth,0) -- cycle ;
		\end{tikzpicture}
	};
	\draw (current page.north west) rectangle (current page.south east);
\end{tikzpicture}
}

% % up 2/3 prosojnica: crno

\usepackage[small, nohug, heads=vee]{diagrams}
\diagramstyle[labelstyle=\scriptstyle]

%\setbeamercovered{invisible} %default
\setbeamercovered{transparent}
%\setbeamercovered{dynamic}


%\usepackage[dvipsnames]{color}

\usepackage[normalem]{ulem} % za strikeout (prečrtat besedo)
% na primer : \sout{Hello World}

% paket, ki ga rabimo za risanje
%\usepackage{tikz}

%\usepackage{lmodern}                            % to get rid of font warnings
%\renewcommand\textbullet{\ensuremath{\bullet}}  % to get rid of font warnings

% podatki
\title{Podatkovni tipi kot začetne algebre in končne koalgebre}
\author{Boštjan Gec}
\institute{mentor: prof. dr. Simpson Alex}

% tvoj izbran stil predstavitve
%\usetheme{Singapore}
\usetheme{Luebeck}
%\usecolortheme{crane}
%\usecolortheme{red}

%themes: http://deic.uab.es/~iblanes/beamer_gallery/index_by_theme.html
	% Antibes sq -L, Berlin sq . . , Copenhagen w, Darmstadt ... , Frankfurt ... , 
	% Ilmenau . ., JuanLesPins L, Luebeck sq w, Warsaw w
	% + others with changed colortheme!
% wins: Berlin or Luebeck or Darmstadt or Frankfurt

% DIY  % % % % % % % % % % % % % % % % % % % % % 
%colors:
%\documentclass[usenames,dvipsnames]{beamer}
\usepackage{color}
%\usepackage[usenames,dvipsnames,svgnames,table]{xcolor}
%\setbeamercolor{structure}{fg=RawSienna}
%\setbeamercolor{structure}{fg=Brown}
\setbeamercolor{structure}{fg=Bittersweet}
%\setbeamercolor{structure}{fg=ForestGreen}
%\setbeamercolor{structure}{fg=OliveGreen}
%\setbeamercolor{structure}{fg=Mahogany}
%\setbeamercolor{structure}{fg=Sepia}
%\setbeamercolor{structure}{fg=Orange}
%\setbeamercolor{structure}{fg=BurntOrange}
%\setbeamercolor{structure}{fg=Maroon}
%\setbeamercolor{structure}{fg=BrickRed}


%\useoutertheme{smoothtree}
%\useinnertheme{rectangles}

%\setbeamercolor{alerted text}{fg=orange}
%\setbeamercolor{background canvas}{bg=orange}
%\setbeamercolor{block body alerted}{bg=normal text.bg!90!black}
%\setbeamercolor{block body}{bg=normal text.bg!90!black}
%\setbeamercolor{block body example}{bg=normal text.bg!90!black}
%\setbeamercolor{block title alerted}{use={normal text,alerted text},fg=alerted text.fg!75!normal text.fg,bg=normal text.bg!75!black}
%\setbeamercolor{block title}{bg=OliveGreen!80!black}
%\setbeamercolor{block title example}{use={normal text,example text},fg=example text.fg!75!normal text.fg,bg=normal text.bg!75!black}
%\setbeamercolor{fine separation line}{}
%\setbeamercolor{frametitle}{bg=OliveGreen, fg=white}
%\setbeamercolor{item projected}{fg=brown}
%\setbeamercolor{normal text}{bg=black,fg=orange} % <--- very good!
%\setbeamercolor{palette sidebar primary}{bg=orange, use=normal text,fg=brown}
%\setbeamercolor{palette sidebar quaternary}{use=structure,fg=structure.fg}
%\setbeamercolor{palette sidebar secondary}{use=structure,fg=structure.fg}
%\setbeamercolor{palette sidebar tertiary}{use=normal text,fg=normal text.fg}
%\setbeamercolor{section in sidebar}{fg=brown}
%\setbeamercolor{section in sidebar shaded}{fg=orange}
%\setbeamercolor{separation line}{}
%\setbeamercolor{sidebar}{bg=red, fg=brown}
%\setbeamercolor{sidebar}{parent=palette primary}
%\setbeamercolor{structure}{fg=Bittersweet}

%\setbeamercolor{subsection in sidebar}{fg=brown}
%\setbeamercolor{subsection in sidebar shaded}{fg=grey}
%\setbeamercolor{title}{bg=OliveGreen}
%\setbeamercolor{titlelike}{bg=OliveGreen}
%
%\setbeamertemplate{blocks}[circles]
%%\setbeamertemplate{blocks}[rectangles][shadow=true]
%\setbeamertemplate{background canvas}[vertical shading][bottom=white,top=structure.fg!25]
%\setbeamertemplate{sidebar canvas left}[horizontal shading][left=white!40!black,right=black]
% % % % % % % % % % % % % % % % % % % % % % % % % % % % % %


\begin{document}


\begin{frame}
  \maketitle
%  \fillblack{0.33}
\end{frame}

%\begin{frame}
%	\begin{itemize}
%		\item
%	\end{itemize}
%\end{frame}

\section{Uvod}

\begin{frame}{Uvod}
	\begin{itemize}
		\item V programiranju so seznami in drevesa pomembne (in uporabne) podatkovne strukture.
		
		\item V modernih programskih jezikih (npr. jaz bom uporabljal Haskell) lahko skonstruiramo podatkovne tipe za sezname in drevesa.
		
		\item Te podatkovne tipe lahko karakteriziramo z za"cetnimi algebrami in tudi kon"cnimi koalgebrami.
		
		\item Ta karakterizacija nam daje metode in na"cine pri programiranju s seznami in drevesi.
	\end{itemize}
\end{frame}
\begin{frame}{Za"cetne algebre in kon"cne koalgebre}
	\begin{itemize}
		\item 	Za"cetna algebra in kon"cna koalgebra sta koncepta iz teorije kategorij.
		
		\item 
		
		\item V predstavitvi bom namesto splo"sne definicije predstavil njeno bolj zo"zeno verzijo na enem pomembnem primeru.
		
		\item Izbran primer: seznami v programiranju in njihov analog v matematiki: zaporedja.
		
		\item 
	\end{itemize}
\end{frame}

\begin{frame}{Seznami}
	Nekaj primerov seznamov:
	\begin{itemize}
		\item	 \text{[1, 2, 3, 4, 5]}
		\item	\text{[1, 3, 5]}
		\item	\text{[  \ ]}
	\end{itemize}
	
	... [Int]	\\
	Seznami razli"cnih tipov:
	\begin{itemize}
		\item	$[\frac{1}{2},\frac{2}{3},\frac{10}{7}]$ ... \text{[Fractional]}
		\item	\text{[0.01, pi, e, 2.04]}  ... \text{[Floating]}
		\item	\text{['p', 'i', 'e'] }      ... \text{[Char]}
		\item	\text{[ [0,1,2,3], [\ \  ], [3] ]}  ... \text{[[Int]]}
	\end{itemize}	
\end{frame}
\begin{frame}{Seznami v matematiki}
	Nekaj primerov kon"cnih zaporedij:
	\begin{itemize}
		\item	\text{1, 2, 3, 4, 5 ...  oznaka na dana"snji} \text{predstavitvi bo 1 : 2 : 3 : 4 : 5}
		\item	\text{1, 3, 5 ... oznana"cimo z 1 : 3 : 5}
		\item	$\varepsilon$ ... oznaka za prazno zaporedje v teoreti"cnem ra"cunalni"stvu
	\end{itemize}
	X ... mno"zica \\
	$X^\star$ ... oznaka za mno"zico vseh kon"cnih zaporedij v teoreti"cnem ra"cunalni"stvu
	
\end{frame}

\section{Matematika}

\begin{frame}{Matematika}
	
		
	\begin{itemize}
		\item Sedaj "zelimo pokazati, da so mno"zica $X^\star$ kon"cnih zaporedij in njim podobni podatkovni tipi (seznami) karakterizirani kot za"cetne algebre.
		
		\item Koncept za"cetne algebre je smiselen za vsak funktor na kategoriji.
		
		\item Mi si bomo ogledali samo konkreten primer definicije, ki nas tu zanima.
	\end{itemize}
\end{frame}
\begin{frame}{Definicije}
	\begin{block}{Definicija}
		Naj bosta X in A mno"zici.\\
		$$ 1 + X \times  A   \stackrel{\mathrm{def}}{=} \{ * \} \cup \{ (x,a) \mid x \in X , a \in A \}  $$ \\
		kjer je $*$ nek izbran element, ki ni par.
	\end{block}
	
	%%	\pause
	%	pause
	\begin{block}{O"citno}
		$$ | 1 + X \times A | = 1 + | X \times  A | = 1 + | X | \times | A | $$
	\end{block}
	
\end{frame}


\begin{frame}
	
	\begin{block}{Predpostavka}
		
	Naj bo X mno"zica. \\
	Operacija F s predpisom 
%	$ F: A \mapsto 1+X\times A $
	$ F(A) = 1+X\times A $
	, tj. da poljubni mno"zici $A$ priredi mno"zico $1+X\times A$, je \underline{\textbf{funktor na primeru seznamov}}, to je za vsako preslikavo $ f: A \to B $ obstaja preslikava $\overline{f}: F(A) \to F(B) $ definirana z
	$$ \overline{f}(z) = 
	\left\{ \begin{array}{cll}
	* & ; & z = * \\
	(x, f(a)) & ; & z = (x, a)
	\end{array} \right. $$ 
		in zado"s"ca pogojema 
	$$ \begin{array}{clll}
%	\overline{id_A} & = & id_{\,1\!+\!X\!\times \!A}  & in \\
	\overline{id_A} & = & id_{F(A)}  & in \\
	\overline{g \circ f} & = & \overline{g} \circ \overline{f} & .
	\end{array}   $$ 
	
	\end{block}

\end{frame}

\begin{frame}
	
	
	\begin{block}{Definicija}
	Naj bosta X in  $\mathrm{I}$ mno"zici in $i$ preslikava $1+X \times \mathrm{I} \to \mathrm{I}$ . \\
	Par $(\mathnormal{I},\mathnormal{i}) $ je \underline{\textbf{za"cetna algebra na primeru seznamov}}, "ce za vsako mno"zico A in preslikavo $f: 1+X \times A \to A$ obstaja enoli"cno dolo"cena preslikava $ \mathnormal{h} : \mathrm{I} \to A $, tako da velja 
	$ h \circ i = f \circ \overline{h} $,
	oz. da spoden graf komutira.	
	$$	\begin{diagram}
	1+X \times \mathrm{I} & & \rTo^{\overline{h}} & & 1+X \times A \\
	\dTo_{\mathit{i}} & & & & \dTo_{\mathnormal{f}} \\
	\textrm{I} & & \rTo^{h} & & A
	\end{diagram} $$
	
	\end{block}
	
\end{frame}

\begin{frame}{Trditev}

	\begin{block}{Definicija}
	Naj bo $\mathnormal{X}^\star$ mno"zica vseh kon"cnih zaporedij sestavljenih iz elementov iz X.	
	\end{block}
	\begin{block}{Definicija}
	Naj bo preslikava $i: 1+\mathnormal{X}\times \mathnormal{X}^\star \to \mathnormal{X}^\star$ definirana kot
	$$ \mathnormal{i}(z) \stackrel{\mathrm{def}}{=} \left\{ \begin{array}{lll}
	\varepsilon & ; & z = * \\
	x : y & ; & z = (x,y)
	\end{array} \right. . $$ 
	\end{block}
	
	\begin{block}{Trditev}
	  Par $( \mathnormal{X^\star} ,i)$ je \underline{za"cetna algebra na primeru seznamov}.
		
	\end{block}
		 
%	\fillblack{0.33}
\end{frame}

\begin{frame}{Haskell}
%		\definecolor{softyellow}{rgb}{0.98,0.98,0.75}
%		\setbeamercolor{loweryel}{fg=black,bg=softyellow}


\begin{block}{Matemati"cno}
	$$ 1+\Z \times A $$
	
\end{block}

Haskell::


\begin{beamerboxesrounded}[lower=loweryel,shadow=false]{}
	\texttt{data Seznam\_funktor a = Zvezdica | Par Int a}
\end{beamerboxesrounded}


\begin{block}{\underline{Funktor na primeru seznamov}}
	Naj bo $X = \Z$.
	$$ F(A) = 1+\Z \times A $$
	
\end{block}



	
\end{frame}



\begin{frame}

\begin{block}{Matemati"cno}
	Iz definicije \underline{funktorja na primeru seznamov}:
		$$ \overline{f}(z) = 
		\left\{ \begin{array}{cll}
		* & ; & z = * \\
		(x, f(a)) & ; & z = (x, a)
		\end{array} \right. $$ 
	
\end{block}	
	V Haskell-u preslikava
	%	$ \bar{\textcolor{white}{h} }$ 
	$ \overline{\textcolor{white}{h} } $ 
%	$ \overline{\textcolor{gray}{h} } $ 
%	%	( v primeru $ \overline{h} ) $ 
	tipa \\
	\texttt{::(a -> b) -> (Seznam\_funktor a -> Seznam\_funktor b)} \\
%	lakho definiramo tako:	
	\begin{beamerboxesrounded}[lower=loweryel,shadow=false]{}
		\texttt{prečno f Zvezdica = Zvezdica} \\
		\texttt{prečno f (Par x a)= Par x (f a)}
	\end{beamerboxesrounded}
\end{frame}

\section{Haskell}

\begin{frame}
	\begin{block}{Matematika}
		V matematiki imamo enakost:
		$$\Z^\star  \stackrel{\mathrm{def}}{=} \{\varepsilon\} \cup \{ x:y \mid x \in \Z,  y \in \Z^\star   \}$$
	\end{block}
	Ekvivalentna je tisti v Haskellu:
	\begin{beamerboxesrounded}[lower=loweryel,shadow=false]{}
	\texttt{data Seznam = Prazen | Sestavi Int Seznam}
	\end{beamerboxesrounded}
	Tako je definirano v ra"cunalni"stvu.
	\\
	Primer: \\
	\texttt{Sestavi 1 (Sestavi 3 (Sestavi 5 Prazen))} \\
	 = [1,3,5]

\end{frame}

\begin{frame}
	\begin{block}{Matemati"cno}	
		Naj bo $X = \Z$. \\
		V trditvi je $i: 1\!+\!\Z\! \times\! \Z^\star \to \Z^\star$ definirana kot
		$$ \mathnormal{i}(z) \stackrel{\mathrm{def}}{=} \left\{ \begin{array}{lll}
		\varepsilon & ; & z = * \\
		x : y & ; & z = (x,y)
		\end{array} \right. . $$ 
	\end{block}
	
	V Haskell-u funkcija tipa \\
	\texttt{i::Seznam\_funktor Seznam -> Seznam} \\	
	\begin{beamerboxesrounded}[lower=loweryel,shadow=false]{}
		\texttt{i Zvezdica = Prazen} \\
		\texttt{i (Par x y)= Sestavi x y}
	\end{beamerboxesrounded}
		
\end{frame}


\begin{frame}
	\begin{block}{Matemati"cno}
		Iz dokaza:
		$$ \mathnormal{h}(z) \stackrel{\mathrm{def}}{=} \left\{ \begin{array}{lll}
		f(*) & ; & z = \varepsilon \\
		f(x,h(y)) & ; & z = x:y
		\end{array} \right. . $$	
	\end{block}
	
	V Haskell-u funkcija tipa \\
	\texttt{zlo"zi::(Seznam\_funktor a -> a) -> (Seznam -> a)} \\	
	\begin{beamerboxesrounded}[lower=loweryel,shadow=false]{}
		\texttt{zlo"zi f Prazen = f Zvezdica} \\
		\texttt{zloži f (Sestavi x y) = f (Par x (zloži f y))}
	\end{beamerboxesrounded}
	
To funkcijo imenujemo tudi \underline{lastnost za"cetne algebre}. 
V angle"s"cini fold.\\

Funkcija \texttt{zlo"zi f} ustreza $h$ .

\end{frame}


\begin{frame}{Primer}
	Primeri programov definiranih s pomo"cjo lastnosti za"c. algebre:
	\begin{itemize}
		\item \texttt{sum}
	\end{itemize}
	
\begin{beamerboxesrounded}[lower=loweryel,shadow=false]{}
	\texttt{ sum [1,3,5] } 
\end{beamerboxesrounded}
	nam vrne 9 .
	
\end{frame}


\begin{frame}
	
	Definirajmo $ f :  1+\Z \times \Z \to \Z $ tako:
	$$ f(x) = \left\{ \begin{array}{cll}
	0 & ; & x = * \\
	m+n & ; & x = (m,n)
	\end{array} \right. $$
	V Haskellu:
	\begin{beamerboxesrounded}[lower=loweryel,shadow=false]{}
		\texttt{f Zvezdica = 0 \\
			f (Par m n) = m + n} 
	\end{beamerboxesrounded}
	
	Izkaze se, da izraz	\\
	\begin{beamerboxesrounded}[lower=loweryel,shadow=false]{}
		\texttt{zloži f} 
	\end{beamerboxesrounded}
	uztreza funkciji \texttt{sum }.
	
	\begin{beamerboxesrounded}[lower=loweryel,shadow=false]{}
		\texttt{zloži f (Sestavi 1 (Sestavi 3 (Sestavi 5 Prazen))) } 
	\end{beamerboxesrounded}
	nam vrne 9 . Kar je isto kot 
	\texttt{sum [1,3,5]} .
	
	
\end{frame}


		
		
%     $$$    empty frame    $$$
%		\begin{frame}{Trditev}
%			def zac alg na primeru seznamov, opomba, trditev
%			\begin{block}{Trditev}
%			\end{block}
%		\end{frame}
		
		
%------------------------------------------
\begin{frame}{Literatura}


%{
%	\tiny
	\begin{thebibliography}{99}
%		\bibitem{krivine} T. Kaczorek, Polynomial and Rational Matrices, Springer 2007
		\bibitem{krivine}  Brent Yorgey, Introduction to Haskell,
%		 http://www.cis.upenn.edu/$\sim$cis194/spring13/, 
		pomlad 2013
		
		\bibitem{krivine} Brian Whetter, A Gentle Introduction to Category Theory,
%		http://www.math.ups.edu/~bryans/Current/Spring\_2014/10\_Brian\_Abstract\_Project.pdf, 
		april 2014
		
		\bibitem{krivine} D. Sangiorgi and J. Rutten, Advanced topics in bisimulation and coinduction, pp. 38-99, Cambridge Tracts in Theoretical Computer Science Volume 52, Cambridge University Press, 2011. 
		
%		http://homepages.cwi.nl/~janr/papers/
%http://www.cambridge.org/si/academic/subjects/computer-science/distributed-networked-and-mobile-computing/advanced-topics-bisimulation-and-coinduction?format=HB&isbn=9781107004979

		\bibitem{krivine} Peter Smith, Cathegory theory a gentle intro,
%		 http://www.logicmatters.net/resources/pdfs/GentleIntro.pdf, 
		februar 2016


		
		
		
	\end{thebibliography}
%}
\end{frame}

\end{document}

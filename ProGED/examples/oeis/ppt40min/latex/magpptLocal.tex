\documentclass[t,usenames,dvipsnames]{beamer} % [t] pomeni poravnavo na vrh slida

% \usepackage{etex} % vključi ta paket, če ti javi napako, da imaš naloženih preveč paketov.

% standardni paketi
\usepackage[slovene]{babel}
\usepackage[T1]{fontenc}
\usepackage[utf8]{inputenc}
\usepackage{amssymb}
\usepackage{amsmath}
% \usepackage{diagrams}
	\usepackage{forest}
\forestset{
dg edges/.style={for tree={parent anchor=south, child anchor=north,align=center,base=bottom,where n children=0{tier=word,edge=dotted,calign with current edge}{}}},
dgii edges/.style={for tree={parent anchor=south, child anchor=north,base=bottom}},
}

\newcommand{\pgrammar}{
    R :  \\
    S $\to$ Povedek Predmet [0.3] \\
    S $\to$ Povedek Predmet Prislovno\_dolocilo\_nacina [0.7] \\
    Prislovno\_dolocilo\_nacina $\to$ s Predmet [1] \\
    Povedek $\to$ vidim [0.9] \\
    Povedek $\to$ dokazujem [0.1] \\
    Predmet $\to$ Predmet s Predmet [0.2] \\
    Predmet $\to$ psa [0.3] \\
    Predmet $\to$ teleskopom [0.5]
}
\newcommand{\grammar}{
      \only<2>{R :=  \\
    S $\to$ Povedek Predmet \\
    S $\to$ Povedek Predmet Prislovno\_dolocilo\_nacina [0.7] \\
    Prislovno\_dolocilo\_nacina $\to$ s Predmet [1] \\
    Povedek $\to$ vidim [0.9] \\
    Povedek $\to$ dokazujem [0.1] \\
    Predmet $\to$ Predmet s Predmet [0.2] \\
    Predmet $\to$ psa [0.3] \\
    Predmet $\to$ teleskopom [0.5]} 
}
% % down 4.4.2017
\newcommand{\R}{\mathbb R}
\newcommand{\N}{\mathbb N}
\newcommand{\Z}{\mathbb Z}
\newcommand{\C}{\mathbb C}
\newcommand{\Q}{\mathbb Q}
\definecolor{softyellow}{rgb}{0.98,0.98,0.75}
\setbeamercolor{loweryel}{fg=black,bg=softyellow}
% % up 4.4.2017

% % down 2/3 prosojnica: crno
% \usepackage{tikz}
% \newcommand{\fillblack}[1]{
% 	\begin{tikzpicture}[remember picture, overlay]
% 	\node [shift={(0 cm,0cm)}]  at (current page.south west)
% 	{%
% 		\begin{tikzpicture}[remember picture, overlay] at (current page.south west)
% 		\draw [fill=black] (0, 0) -- (0,#1 \paperheight) --
% 		(\paperwidth,#1 \paperheight) -- (\paperwidth,0) -- cycle ;
% 		\end{tikzpicture}
% 	};
% 	\draw (current page.north west) rectangle (current page.south east);
% \end{tikzpicture}
% }

% % up 2/3 prosojnica: crno

% \usepackage[small, nohug, heads=vee]{diagrams}
% \diagramstyle[labelstyle=\scriptstyle]

%\setbeamercovered{invisible} %default
\setbeamercovered{transparent}
%\setbeamercovered{dynamic}


%\usepackage[dvipsnames]{color}

\usepackage[normalem]{ulem} % za strikeout (prečrtat besedo)
% na primer : \sout{Hello World}

% paket, ki ga rabimo za risanje
%\usepackage{tikz}

%\usepackage{lmodern}                            % to get rid of font warnings
%\renewcommand\textbullet{\ensuremath{\bullet}}  % to get rid of font warnings

% podatki
\title{Odkrivanje enačb za celoštevilska zaporedja z verjetnostnimi gramatikami}
\author{Boštjan Gec}
\institute{mentor: prof. dr. Ljupčo Todorovski}

% tvoj izbran stil predstavitve
%\usetheme{Singapore}
\usetheme{Luebeck}
%\usecolortheme{crane}
%\usecolortheme{red}

%themes: http://deic.uab.es/~iblanes/beamer_gallery/index_by_theme.html
	% Antibes sq -L, Berlin sq . . , Copenhagen w, Darmstadt ... , Frankfurt ... ,
	% Ilmenau . ., JuanLesPins L, Luebeck sq w, Warsaw w
	% + others with changed colortheme!
% wins: Berlin or Luebeck or Darmstadt or Frankfurt

% DIY  % % % % % % % % % % % % % % % % % % % % %
%colors:
%\documentclass[usenames,dvipsnames]{beamer}
\usepackage{color}
%\usepackage[usenames,dvipsnames,svgnames,table]{xcolor}
% \setbeamercolor{structure}{fg=RawSienna}
% \setbeamercolor{structure}{fg=Brown}
\setbeamercolor{structure}{fg=Bittersweet}
\setbeamercolor{structure}{fg=ForestGreen}
% \setbeamercolor{structure}{fg=OliveGreen}
% \setbeamercolor{structure}{fg=Mahogany}
% \setbeamercolor{structure}{fg=Sepia}
% \setbeamercolor{structure}{fg=Orange}
% \setbeamercolor{structure}{fg=BurntOrange}
% \setbeamercolor{structure}{fg=Maroon}
% \setbeamercolor{structure}{fg=BrickRed}


%\useoutertheme{smoothtree}
%\useinnertheme{rectangles}

%\setbeamercolor{alerted text}{fg=orange}
%\setbeamercolor{background canvas}{bg=orange}
%\setbeamercolor{block body alerted}{bg=normal text.bg!90!black}
%\setbeamercolor{block body}{bg=normal text.bg!90!black}
%\setbeamercolor{block body example}{bg=normal text.bg!90!black}
%\setbeamercolor{block title alerted}{use={normal text,alerted text},fg=alerted text.fg!75!normal text.fg,bg=normal text.bg!75!black}
%\setbeamercolor{block title}{bg=OliveGreen!80!black}
%\setbeamercolor{block title example}{use={normal text,example text},fg=example text.fg!75!normal text.fg,bg=normal text.bg!75!black}
%\setbeamercolor{fine separation line}{}
%\setbeamercolor{frametitle}{bg=OliveGreen, fg=white}
%\setbeamercolor{item projected}{fg=brown}
%\setbeamercolor{normal text}{bg=black,fg=orange} % <--- very good!
%\setbeamercolor{palette sidebar primary}{bg=orange, use=normal text,fg=brown}
%\setbeamercolor{palette sidebar quaternary}{use=structure,fg=structure.fg}
%\setbeamercolor{palette sidebar secondary}{use=structure,fg=structure.fg}
%\setbeamercolor{palette sidebar tertiary}{use=normal text,fg=normal text.fg}
%\setbeamercolor{section in sidebar}{fg=brown}
%\setbeamercolor{section in sidebar shaded}{fg=orange}
%\setbeamercolor{separation line}{}
%\setbeamercolor{sidebar}{bg=red, fg=brown}
%\setbeamercolor{sidebar}{parent=palette primary}
%\setbeamercolor{structure}{fg=Bittersweet}

%\setbeamercolor{subsection in sidebar}{fg=brown}
%\setbeamercolor{subsection in sidebar shaded}{fg=grey}
%\setbeamercolor{title}{bg=OliveGreen}
%\setbeamercolor{titlelike}{bg=OliveGreen}
%
%\setbeamertemplate{blocks}[circles]
%%\setbeamertemplate{blocks}[rectangles][shadow=true]
%\setbeamertemplate{background canvas}[vertical shading][bottom=white,top=structure.fg!25]
%\setbeamertemplate{sidebar canvas left}[horizontal shading][left=white!40!black,right=black]
% % % % % % % % % % % % % % % % % % % % % % % % % % % % % %


\begin{document}


\begin{frame}
  \maketitle
%  \fillblack{0.33}
\end{frame}


\section{Uvod}

\begin{frame}{Uvod}
	\begin{block}{Odkrivanje enačb za celoštevilska zaporedja z verjetnostnimi gramatikami}
	\begin{itemize}
		\item Odkrivanje ena"cb
		\item Verjetnostne gramatike
		\item Celo"stevilska zaporedja
	\end{itemize}
	\end{block}
\end{frame}

% \begin{frame}{Algoritem za odkrivanje ena"cb}
% % \frametitle{Naslov prosojnice}
% \framesubtitle{Linearna regresija}
% % 	\begin{block}{Odkrivanje enačb za celoštevilska zaporedja z verjetnostnimi gramatikami}
% \pause
% \invisible<1>{Trije koraki:}
% \pause
% 	\begin{enumerate}[<+->]
% 		\item Ena"cba \(y _i - c_0 + c_1\cdot x_{i1}  +
% 		c_2\cdot x_{i2}+ \cdots + c_n\cdot x_{in}  \)
% 	\item Napaka modela \[ \sum_{i=1} ^m (y _i - c_0 + c_1\cdot x_{i1}  +
% 		c_2\cdot x_{i2}+ \cdots + c_n\cdot x_{in} )^2
% 		\]
% kjer je $n+1$ "stevilo stolpcev, m "stevilo vrstic, $y_i$ element matrike X v i-ti vrstici
% in $(n+1)$-tem stolpcu, $x_{ij}$ element matrike X v i-ti vrstici in j-tem stolpcu.\\
% 		X = [x|y]
% 		\item Optimizacija konstant \( c_0, c_1, ..., c_n \), da bo
% 		napaka modela "cim manj"sa.
% 	\end{enumerate}
% \end{frame}

% \begin{frame}{Algoritem za odkrivanje ena"cb}
% \framesubtitle{Odkrivanje ena"cb}
% Podobni trije koraki:
% 	\begin{enumerate}[<+->]
% 	\item Generira seznam ena"cb (izrazov).
% 	\item Napaka modela \[ \sum_{i=1} ^m (y _i - \textcolor{red}{\textbf{izraz}}(c_0, \cdots,  c_n, x_{i1}, \cdots, x_{in}))^2  \]
% kjer so \( y_i, c_j\ in\ x_{ij} \)
% definirani enako kot prej.
% 	\item Optimizacija konstant \( c_0, c_1, \cdots, c_n \).
% 	\end{enumerate}
% \end{frame}

% \begin{frame}{Kontekstno-neodvisna Gramatika}

% 	\begin{block}{Definicija}
% 	Kontekstno-neodvisna gramatika je "cetverica, ki jo ozna"cimo z (N,T,S,R), pri "cemer je:
% 	\begin{itemize}
% 	    \item T mno"zica vseh t.i. terminalnih simbolov
% 	    \item N mno"zica vseh t.i. neterminalnih simbolov
% 	    \item S za"cetni simbol
% 	    \item R mno"zica vseh produkcijskih pravil
% 	    \item N, T in R so kon"cne, za"cetni simbol S je znotraj N, presek N in T prazen.
% 	    R je mno"zica parov \((A, \alpha)\), kjer \( A \in N \) in \( \alpha \in (N \cup T)^* \).
% 	\end{itemize}
%     \end{block}
% \end{frame}

% \begin{frame}{Kontekstno-neodvisna gramatika}
% \subtitle{Drevo izpeljave}
% 	\begin{block}{Definicija (Prepisovalna relacija)}
% 	Prepisovalna relacija \( \Rightarrow_G \) na \( (N \cup T)^* \):
% \[ \beta A \gamma \Rightarrow_G \beta \alpha \gamma, \]
% "ce je \[ (A, \alpha) \in R, \]
% kjer \( \beta, \gamma \in (N \cup T)^*. \)
% 	\end{block}
% 	\pause
% 	\begin{block}{Navada}
% 	     Pi"semo \(A \to \alpha\) namesto \((A, \alpha) \in R \) .
%     \end{block}
% \end{frame}

% \begin{frame}
%     T := \{vidim, psa, s, teleskopom, dokazujem\}

%     N := \{S, Prislovno\_dolocilo\_nacina,  Povedek,
%     Predmet\} \\
%     R := \{ \\
%     S $\to$ Povedek Predmet \\
%     S $\to$ Povedek Predmet Prislovno\_dolocilo\_nacina \\
%     Prislovno\_dolocilo\_nacina $\to$ s Predmet \\
%     Predmet $\to$ Predmet s Predmet \\
%     Povedek $\to$ vidim \\
%     Povedek $\to$ dokazujem \\
%     Predmet $\to$ psa \\
%     Predmet $\to$ teleskopom \\
%     \}
%     \invisible

%     \begin{block}{Trdimo}
%     Stavek:
%     \[ \texttt{vidim psa s teleskopom} \]
%     je element jezika te gramatike.
%     \end{block}

% \end{frame}

% % \begin{frame}
% % dsa
% %	$ h \circ i = f \circ \overline{h} $,
% %	oz. da spoden graf komutira.
% % 	$$	\begin{diagram}
% % 	1+X \times \mathrm{I} & & \rTo^{\overline{h}} & & 1+X \times A \\
% % 	\dTo_{\mathit{i}} & & & & \dTo_{\mathnormal{f}} \\
% % 	\textrm{I} & & \rTo^{h} & & A
% % 	\end{diagram} $$

% % \begin{forest}
% % dg edges
% % [V
% %   [N
% %     [D [the] ]
% %      [child] ]
% %   [reads]
% %   [N
% %     [D [a] ]
% %     [book] ] ]
% % \end{forest}
% % S -> - Povedek -> vidim
% %     `- Predmet -> psa
% %     `- Prislovno_dolocilo_nacina
% % 	`- s
% %         `- Predmet -> teleskopom 		 \ Predmet -> teleskopom
% % \end{frame}

% \begin{frame}
% \begin{forest}
% dg edges
% [S
%   [Povedek
%     [vidim]
%   ]
%   [Predmet
%     [psa]
%   ]
%   [Prislovno\_Dolocilo\_Nacina
%     [s]
%     [Predmet [teleskopom]]
%   ]
% ]
% \end{forest}

% \vspace{1.5cm}
% \footnotesize

%     R : \\
%     \begin{tabular}{l}
%     S $\to$ Povedek Predmet \\
%     S $\to$ Povedek Predmet Prislovno\_dolocilo\_nacina \\
%     Prislovno\_dolocilo\_nacina $\to$ s Predmet \\
%     Predmet $\to$ Predmet s Predmet \\
%     \end{tabular}
%     \begin{tabular}{lllll}
%     Povedek $\to$ vidim &&&&
%     Povedek $\to$ dokazujem \\
%     Predmet $\to$ psa &&&&
%     Predmet $\to$ teleskopom \\
%     \end{tabular}

% \end{frame}

% \begin{frame}
% \begin{forest}
% dgii edges
% [S, red
%   [Povedek, red, edge={red}]
%   [Predmet, red, edge={red}]
%   [Prislovno\_Dolocilo\_Nacina, red, edge={red}]
% ]
% \end{forest}

% \vspace{1.5cm}
% \footnotesize

%     R : \\
%     \begin{tabular}{l}
%     S $\to$ Povedek Predmet \\
%     \textcolor{red}{S $\to$ Povedek Predmet Prislovno\_dolocilo\_nacina} \\
%     Prislovno\_dolocilo\_nacina $\to$ s Predmet \\
%     Predmet $\to$ Predmet s Predmet \\
%     \end{tabular}
%     \begin{tabular}{lllll}
%     Povedek $\to$ vidim &&&&
%     Povedek $\to$ dokazujem \\
%     Predmet $\to$ psa &&&&
%     Predmet $\to$ teleskopom \\
%     \end{tabular}

% \end{frame}

% \begin{frame}
% \begin{forest}
% dgii edges
% [S
%   [ Povedek, red [vidim, red, edge={red}] ]
%   [Predmet]
%   [Prislovno\_Dolocilo\_Nacina]
% ]
% \end{forest}

% \vspace{1.5cm}
% \footnotesize

%     R : \\
%     \begin{tabular}{l}
%     S $\to$ Povedek Predmet \\
%     S $\to$ Povedek Predmet Prislovno\_dolocilo\_nacina \\
%     Prislovno\_dolocilo\_nacina $\to$ s Predmet \\
%     Predmet $\to$ Predmet s Predmet \\
%     \end{tabular}
%     \begin{tabular}{lllll}
%     \textcolor{red}{Povedek $\to$ vidim} &&&&
%     Povedek $\to$ dokazujem \\
%     Predmet $\to$ psa &&&&
%     Predmet $\to$ teleskopom \\
%     \end{tabular}

% \end{frame}

% \begin{frame}
% \begin{forest}
% dgii edges
% [S
%   [ Povedek [vidim] ]
%   [ Predmet, red [psa, red, edge={red}] ]
%   [Prislovno\_Dolocilo\_Nacina]
% ]
% \end{forest}

% \vspace{1.5cm}
% \footnotesize

%     R : \\
%     \begin{tabular}{l}
%     S $\to$ Povedek Predmet \\
%     S $\to$ Povedek Predmet Prislovno\_dolocilo\_nacina \\
%     Prislovno\_dolocilo\_nacina $\to$ s Predmet \\
%     Predmet $\to$ Predmet s Predmet \\
%     \end{tabular}
%     \begin{tabular}{lllll}
%     Povedek $\to$ vidim &&&&
%     Povedek $\to$ dokazujem \\
%     \textcolor{red}{Predmet $\to$ psa} &&&&
%     Predmet $\to$ teleskopom \\
%     \end{tabular}

% \end{frame}

% \begin{frame}
% \begin{forest}
% dgii edges
% [S
%   [ Povedek [vidim] ]
%   [ Predmet [psa] ]
%   [ Prislovno\_Dolocilo\_Nacina, red
%     % [s, red, edge={red} Predmet, red, edge={red}] ]
%     [s, red, edge={red}]
%     [Predmet, red, edge={red}]
%     ]
%   ]
% ]
% \end{forest}

% \vspace{1.5cm}
% \footnotesize

%     R : \\
%     \begin{tabular}{l}
%     S $\to$ Povedek Predmet \\
%     S $\to$ Povedek Predmet Prislovno\_dolocilo\_nacina \\
%     \textcolor{red}{Prislovno\_dolocilo\_nacina $\to$ s Predmet} \\
%     Predmet $\to$ Predmet s Predmet \\
%     \end{tabular}
%     \begin{tabular}{lllll}
%     Povedek $\to$ vidim &&&&
%     Povedek $\to$ dokazujem \\
%     Predmet $\to$ psa &&&&
%     Predmet $\to$ teleskopom \\
%     \end{tabular}

% \end{frame}

% \begin{frame}
% \begin{forest}
% dgii edges
% [S
%   [ Povedek [vidim] ]
%   [ Predmet [psa] ]
%   [ Prislovno\_Dolocilo\_Nacina
%     % [s, red, edge={red} Predmet, red, edge={red}] ]
%     [s]
%     [Predmet, red
%         [teleskopom, red, edge={red}]
%     ]
%     ]
%   ]
% ]
% \end{forest}

% \vspace{1.5cm}
% \footnotesize

%     R : \\
%     \begin{tabular}{l}
%     S $\to$ Povedek Predmet \\
%     S $\to$ Povedek Predmet Prislovno\_dolocilo\_nacina \\
%     Prislovno\_dolocilo\_nacina $\to$ s Predmet \\
%     Predmet $\to$ Predmet s Predmet \\
%     \end{tabular}
%     \begin{tabular}{lllll}
%     Povedek $\to$ vidim &&&&
%     Povedek $\to$ dokazujem \\
%     Predmet $\to$ psa &&&&
%     \textcolor{red}{Predmet $\to$ teleskopom} \\
%     \end{tabular}

% \end{frame}

% \begin{frame}
% \begin{forest}
% dgii edges
% [S
%   [ Povedek [vidim, red] ]
%   [ Predmet [psa, red] ]
%   [ Prislovno\_Dolocilo\_Nacina
%     % [s, red, edge={red} Predmet, red, edge={red}] ]
%     [s, red]
%     [Predmet
%         [teleskopom, red]
%     ]
%     ]
%   ]
% ]
% \end{forest}

% \vspace{1.5cm}
% \footnotesize

%     \textcolor{red}{T := \{vidim, psa, s, teleskopom, dokazujem\}} \\
%     R : \\
%     \begin{tabular}{l}
%     S $\to$ Povedek Predmet \\
%     S $\to$ Povedek Predmet Prislovno\_dolocilo\_nacina \\
%     Prislovno\_dolocilo\_nacina $\to$ s Predmet \\
%     Predmet $\to$ Predmet s Predmet \\
%     \end{tabular}
%     \begin{tabular}{lllll}
%     Povedek $\to$ vidim &&&&
%     Povedek $\to$ dokazujem \\
%     Predmet $\to$ psa &&&&
%     Predmet $\to$ teleskopom \\
%     \end{tabular}

% \end{frame}

% \begin{frame}{Primer: Univerzalna gramatika}

% T := \{'+', '-', '*', '/', 'C', '(', ')', 'sin(', 'cos(', 'sqrt(', 'exp(', 'x1', 'x2', 'x3'\} \\
% N:= \{S, F, T, R, V\} \\


% R: \\
% \begin{tabular}{lllll}
%     S $\to$ S '+' F &&&&
%     S $\to$ S '-' F \\
%     S $\to$ F &&&&
%     F $\to$ F '*' T \\
%     F $\to$ F '/' T &&&&
%     F $\to$ T \\
%     T $\to$ R &&&&
%     T $\to$ 'C' \\
%     T $\to$ V &&&&
%     R $\to$ '(' S ')' \\
%     R $\to$ 'sin(' S ')' &&&&
%     R $\to$ 'cos(' S ')' \\
%     R $\to$ 'sqrt(' S ')' &&&&
%     R $\to$ 'exp(' S ')'  \\
%     V $\to$ 'x1' &&&&
%     V $\to$ 'x2' \\
%     V $\to$ 'x3'
% \end{tabular}

% \begin{block}{Primer stavka v tej gramatiki:}
%   'C' '+' 'C' '*' 'x1' '*' 'exp(' 'x2' '*' 'x2' ')'
% \end{block}

% \end{frame}

% \begin{frame}[plain]
% \begin{columns}[T] % align columns
% \begin{column}{.48\textwidth}
% % \color{red}\rule{\linewidth}{4pt}
% \footnotesize
% \begin{forest}
% dgii edges
% [S
%   [ S
%     [ F [ T [ 'C' ] ] ]
%   ]
%   [ '+' ]
%   [ F
%     [ F [ T [ 'C' ] ] ]
%     [ '*' ]
%     [ T
%         [ F [ T [ V [ 'x1' ] ] ] ]
%         [ '*' ]
%         [ T
%             [ R
%                 [ 'exp(' ]
%                 [ S
%                     [ F
%                         [ F [ T [ V [ 'x2' ] ] ] ]
%                         [ '*' ]
%                         [ T [ V [ 'x2' ] ] ]
%                     ]
%                 ]
%                 [ ')']
%             ]
%         ]
%     ]
%   ]
% ]
% \end{forest}

% \end{column}
% \hfill
% \begin{column}{.48\textwidth}
% \color{blue}\rule{\linewidth}{4pt}
%     S $\to$ S '+' F  \\
%     S $\to$ S '-' F \\
%     S $\to$ F \\
%     F $\to$ F '*' T \\
%     F $\to$ F '/' T \\
%     F $\to$ T \\
%     T $\to$ R  \\
%     T $\to$ 'C' \\
%     T $\to$ V  \\
%     R $\to$ '(' S ')'  \\
%     R $\to$ 'sin(' S ')'  \\
%     R $\to$ 'cos(' S ')'  \\
%     R $\to$ 'sqrt(' S ')' \\
%     R $\to$ 'exp(' S ')'  \\
%     V $\to$ 'x1'  \\
%     V $\to$ 'x2' \\
%     V $\to$ 'x3'  \\

% \end{column}%
% \end{columns}

% \end{frame}

% \begin{frame}[plain]
% \scriptsize
% \begin{forest}
% dg edges
% [S
%   [ S
%     [ F [ T [ 'C' ] ] ]
%   ]
%   [ '+' ]
%   [ F
%     [ F [ T [ 'C' ] ] ]
%         [ '*' ]
%     [ T
%         [ F [ T [ V [ 'x1' ] ] ] ]
%         [ '*' ]
%         [ T
%             [ R
%                 [ 'exp(' ]
%                 [ S
%                     [ F
%                         [ F [ T [ V [ 'x2' ] ] ] ]
%                         [ '*' ]
%                         [ T [ V [ 'x2' ] ] ]
%                     ]
%                 ]
%                 [ ')']
%             ]
%         ]
%     ]
%   ]
% ]
% \end{forest}

% \end{frame}

% \begin{frame}{Algoritem za odkrivanje ena"cb}
% \framesubtitle{Odkrivanje ena"cb}
% Podobni trije koraki:
% 	\begin{enumerate}
% 	\item Generira seznam ena"cb (izrazov).
% 	\item Napaka modela \[ \sum_{i=1} ^m (y _i - \textcolor{red}{\textbf{izraz}(c_0, \cdots,  c_n, x_{i1}, \cdots, x_{in})})^2  \]
% kjer so \( y_i, c_j\ in\ x_{ij} \)
% definirani enako kot prej.
% 	\item Optimizacija konstant \( c_0, c_1, \cdots, c_n \).
% 	\end{enumerate}
% \end{frame}

% \begin{frame}{Algoritem za odkrivanje ena"cb}
% \framesubtitle{Odkrivanje ena"cb}
% Podobni trije koraki:
% 	\begin{enumerate}
% 	\item Generira seznam ena"cb (izrazov).
% 	\item Napaka modela \[ \sum_{i=1} ^m (y _i - \textcolor{red}{\textbf{
%  \( c_0 + c_1 \cdot x_{i1} \cdot exp( x_{i2} ^2 ) \)
% }})^2  \]
% kjer so \( y_i, c_j\ in\ x_{ij} \)
% definirani enako kot prej.
% 	\item Optimizacija konstant \( c_0, c_1, \cdots, c_n \).
% 	\end{enumerate}
% \end{frame}

% \begin{frame}{Prostor ena"cb}
%     \begin{itemize}[<+->]
%         \item S pogo"cjo gramatike generiramo seznam ena"cb.
%         \item Teh je veliko.
%         \item Naivna re"sitev: Samo ena"cbe do dolo"cene vi"sine.
%         \item Druga re"sitev: Spremenimo gramatiko.
% 	\invisible<1-4>{\item \color{red}{Verjetnostne gramatike}}
%     \end{itemize}
% \end{frame}

% \section{Verjetnostne gramatike}

% \begin{frame}{Verjetnostne kontekstno-neodvisne gramatike}
% 	\pause
% 	\begin{block}{Primer od prej:} 
%     T := \{vidim, psa, s, teleskopom, dokazujem\}

%     N := \{S, Prislovno\_dolocilo\_nacina,  Povedek,
%     Predmet\} \\
%     R := \{ \\
% 	S $\to$ Povedek Predmet \only<3>{[0.3]} \\
%     S $\to$ Povedek Predmet Prislovno\_dolocilo\_nacina \only<3>{[0.7]}\\
%     Prislovno\_dolocilo\_nacina $\to$ s Predmet \only<3>{[1]}\\
%     Povedek $\to$ vidim \only<3>{[0.9]}\\
%     Povedek $\to$ dokazujem \only<3>{[0.1]}\\
%     Predmet $\to$ Predmet s Predmet \only<3>{[0.2]}\\
%     Predmet $\to$ psa \only<3>{[0.3]}\\
%     Predmet $\to$ teleskopom \only<3>{[0.5]}\\
%     \}
% 	\end{block}
% \end{frame}

% \begin{frame}
% 	\begin{block}{Pogoj}
% 		\( \forall A \in N:\)
% 		\[ \sum_{ (A\to \alpha) \in R }  p(A \to \alpha) = 1, \]
% 		kjer je \( p: R \to [0,1] \) funkcija,
% 		ki produkcijskemu pravilu priredi njegovo verjetnost.
% 	\end{block}
% \end{frame}

% \begin{frame}
% 	\begin{block}{Verjetost raz"clenitvenega drevesa}
% 		\( \forall \tau \in \Omega: \)
% 		\[ p(\tau) := \prod_{(A \to \alpha) \in R} p(A \to \alpha)^{f(A \to \alpha; \tau)} \]
% 		kjer je \( p(A \to \alpha) \) verjetnost produkcijskega pravila, 
% 		\( f(A\to\alpha) \) frekvenca oz. kratnost pravila \( A\to\alpha \) 
% 		v drevesu $\tau$ in $\Omega$ mno"zica vseh kon"cnih raz"clenitvenih dreves.
% 	\end{block}
% \end{frame}

% \begin{frame}

% 	\only<1>{$ \tau $ := \\
% \begin{forest}
% dgii edges
% [S
%   [Povedek
%     [vidim]
%   ]
%   [Predmet
%     [psa]
%   ]
%   [Prislovno\_Dolocilo\_Nacina
%     [s]
%     [Predmet [teleskopom]]
%   ]
% ]
% \end{forest}}

% 	\footnotesize
% 	\only<2>{R :=  \\
% 	S $\to$ Povedek Predmet [0.3] \\
%     S $\to$ Povedek Predmet Prislovno\_dolocilo\_nacina [0.7] \\
%     Prislovno\_dolocilo\_nacina $\to$ s Predmet [1] \\
%     Povedek $\to$ vidim [0.9] \\
%     Povedek $\to$ dokazujem [0.1] \\
%     Predmet $\to$ Predmet s Predmet [0.2] \\
%     Predmet $\to$ psa [0.3] \\
% Predmet $\to$ teleskopom [0.5]} 

% 	\vspace{0.5cm}   
%     \small

% 	\( p( \tau) = \) \\
% 	\( = p(S\to\ Povedek\ Predmet Prislovno\_dolocilo\_nacina) \cdot \)
% 	\( \cdot p(Povedek\ \to vidim) 
% 	\cdot p(Predmet\ \to psa) \cdot \)
% 	\( \cdot p(Prislovno\_dolocilo\_nacina\ \to s\ Predmet) \cdot \)
% 	\( \cdot p(Predmet \to teleskopom) \) \\
% 	% \( =  0.7 (S) x 1(vidim) x 0.3(psa) x 1 (Prislovno_dol_nacina) = 0.7*0.3=0.21
% 	% \( =  0.7 (S) \cdot 0.9(vidim) \cdot 0.3(psa) \cdot 1 (Prislovno\_dol_nacina) \cdot 0.5(teleskopom) \)
% 	\only<2>{\( =  0.7 \cdot 0.9 \cdot 0.3 \cdot 1 \cdot 0.5 = 0.0945 \)}
% \end{frame}

\begin{frame}{Naklju"cni algoritem}
  \begin{enumerate}
     \item Za"cnemo v $S$
     \item Ponavljamo korak:\\
     	V trenutni stav"cni obliki izberemo prvi neterminal, ki nastopa.
     \item Izberemo pravilo s tem neterminalom.
     \item Naklju"cno izbiramo glede na porazdelitev, podano z pravili
     \item Kon"camo, ko dobimo stavek (ostanejo samo terminali). 
  \end{enumerate}
\end{frame}

\begin{frame}{T.i. verjetnost drevesa}

  Verjetnost, da algoritem vrne drevo $\tau$:
	\( P( \tau ) := P( Algoritem\ vrne\ \tau) : = \) \\
	\( P( Algoritem\ vrne  \)
	\begin{forest}
  dgii edges
  [S
    [Povedek
      [vidim]
    ]
    [Predmet
      [psa]
    ]
    [Prislovno\_Dolocilo\_Nacina
      [s]
      [Predmet [teleskopom]]
    ]
  ]
	\end{forest} 
	) = ? \only<2>{ \( \ne p(\tau) \) }
\end{frame}

\begin{frame}
	\( P( \tau_0 ) := P( Algoritem\ vrne\ \tau_0) : = \) \\
	\( P( Algoritem\ vrne  \)
	\begin{forest}
  dgii edges
  [S
    [Povedek
    ]
    [Predmet
    ]
    [Prislovno\_Dolocilo\_Nacina
    ]
  ]
	\end{forest} 
	) = \only<1>{?} 
	\only<3-4>{ \( 
	 = p(S\to\ Povedek\ Predmet Prislovno\_dolocilo\_nacina) 
	 = 0.7 \) } \\
	 \only<4>{ \[ = p( \tau_0) \] }
		  
  \vspace{0.5cm}
  \footnotesize
	\only<2-3>{\pgrammar}
\end{frame}

\begin{frame}[plain]
	\( P( \tau_1 ) := P( Algoritem\ vrne\ \tau_1) : = \) \\
	\( P( Algoritem\ vrne  \)
	\begin{forest}
  dgii edges
  [S
    [Povedek
      [vidim]
    ]
    [Predmet
    ]
    [Prislovno\_Dolocilo\_Nacina
    ]
  ]
	\end{forest} \\
	) = \only<1,3-5>{?} 
	 \only<2>{\( 0.7\cdot 0.9 \)} 
	 \\
  \vspace{0.5cm}
	\( P(A) = P(A|H_1)\cdot P(H_1) + P(A|H_2)\cdot P(H_2) \), \\
	\only<2>{ \hspace{1.2cm}\( = P(A|H_1)\cdot {\color<2>{red}0.7} + P(A|H_2)\cdot {\color<2>{red}0.3} \), \\ }
	\only<3-5>{ \hspace{1.2cm}\( =  {\color<3>{red}0.9} \cdot 0.7 + {\color<3>{red}0} \cdot 0.3 = 0.63 \), \\ }
	  \only<4-5>{ \( = p(S \to \ Povedek\ Predmet\ Prislovno\_dolocilo\_nacina) \cdot 
	  \cdot p(Povedek \to vidim) \) \\ }
	  \only<5>{ \[ = p(\tau_1) \] }
	  \vspace{0.5cm}
	\only<1-3>{ kjer je \( H_1:= \) \{ dogodek, da v prvem koraku algoritem izbere 
	pravilo  \\ 
	\ \ (S $\to$ Povedek Predmet Prislovno\_dolocilo\_nacina) \}, \\
	\( H_2:= H_1^C \), \\
	A:= \{ dogodek, da v prvem koraku algoritem izbere 
	pravilo (S $\to$ Povedek Predmet Prislovno\_dolocilo\_nacina) 
	in v drugem koraku izbere pravilo (Povedek $\to$ vidim) \} }

\end{frame}


\section{Celo"stevilska zaporedja}
\begin{frame}
\end{frame}

\end{document}

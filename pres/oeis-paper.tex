\documentclass[t,usenames,dvipsnames]{beamer} % [t] pomeni poravnavo na vrh slida

    % \State Randomly choose rewriting rule 
    % \State choose rewriting rule Randomly 
    % relation closure more verbose on slides

% \usepackage{etex} % vključi ta paket, če ti javi napako, da imaš naloženih preveč paketov.
% \usefonttheme[stillsansserifsmall,stillsansseriflarge]{serif}

% \usepackage{amsfonts}
% \usepackage[mathscr]{eucal}
% \usepackage[mathscr]
\usepackage{mathrsfs}


% standardni paketi
\usepackage[slovene]{babel}
\usepackage[T1]{fontenc}
\usepackage[utf8]{inputenc}
\usepackage{amsmath,amssymb,amsfonts,amsthm} % matematični paketi
% \usepackage{amssymb}
% \usepackage{amsmath}
% \usepackage{diagrams}
\usepackage{array}
\usepackage{multirow}
\usepackage{qtree}
	\usepackage{forest}
\forestset{
dg edges/.style={for tree={parent anchor=south, child anchor=north,align=center,base=bottom,where n children=0{tier=word,edge=dotted,calign with current edge}{}}},
dgii edges/.style={for tree={parent anchor=south, child anchor=north,base=bottom}},
}

\newcommand{\pgrammar}{
    R :  \\
    S $\to$ Subject Verb Object [0.3] \\
    S $\to$ Subject Verb Object Manner [0.7] \\
    Manner $\to$ with Object [1] \\
    Verb $\to$ see [0.9] \\
    Verb $\to$ invent [0.1] \\
    Subject $\to$ Subject with Subject [0.2] \\
    Subject $\to$ dog [0.3] \\
    Subject $\to$ telescope [0.5]
}

% \newcommand{\grammar}{
%       \only<2>{R :=  \\
%     S $\to$ Povedek Predmet \\
%     S $\to$ Povedek Predmet Prislovno\_dolocilo\_nacina [0.7] \\
%     Prislovno\_dolocilo\_nacina $\to$ s Predmet [1] \\
%     Povedek $\to$ vidim [0.9] \\
%     Povedek $\to$ dokazujem [0.1] \\
%     Predmet $\to$ Predmet s Predmet [0.2] \\
%     Predmet $\to$ psa [0.3] \\
%     Predmet $\to$ teleskopom [0.5]} 
% }

\usepackage{fp}
% \newcommand{\add}[2]{ \FPeval{ \p }{round(#1+#2, 0)} \p }
\newcommand{\eval}[1]{ \FPeval{ \p }{round(#1, 0)} \p }
% \newcommand{\add}[2]{ \FPeval{ \p }{#2+#1} \p }
% \newcommand{\add}[2][2]{ \FPeval{ \p }{round(#2, #1)} \p }
% \newcommand{\datan}[1]{
% \newcommand{\datatwo}[1]{
%     \( \left[ \begin{array}{ccccccccc}
%         a_1 & a_2 &  a_{\eval{#1+1}} \\
%         a_2 & a_3 &  a_{\eval{#1+2}} \\
%     \vdots & \vdots  & \vdots \\
%         a_{\eval{50-#1}} & a_{\eval{50-(#1-1)}} & a_{50} \\
%     \end{array} \right] \)}
% \newcommand{\dataset}[1]{
%     \( \left[ \begin{array}{ccccccccc}
%         a_1 & a_2 & \cdots & a_{\eval{#1+1}} \\
%         a_2 & a_3 & \cdots & a_{\eval{#1+2}} \\
%     \vdots & \vdots & \ddots & \vdots \\
%         a_{\eval{50-#1}} & a_{\eval{50-(#1-1)}} & \cdots & a_{50} \\
%     \end{array} \right] \)}
% \newcommand{\datatwon}[1]{
%     \( \left[ \begin{array}{ccccccccc}
%         1 & a_1 & a_2 &  a_{\eval{#1+1}} \\
%         2 & a_2 & a_3 &  a_{\eval{#1+2}} \\
%     \vdots & \vdots & \vdots  & \vdots \\
%         \eval{50-#1} & a_{\eval{50-#1}} & a_{\eval{50-(#1-1)}} & a_{50} \\
%     \end{array} \right] \)}
% \newcommand{\datasetn}[1]{
%     \( \left[ \begin{array}{ccccccccc}
%         1 & a_1 & a_2 & \cdots & a_{\eval{#1+1}} \\
%         2 & a_2 & a_3 & \cdots & a_{\eval{#1+2}} \\
%     \vdots & \vdots & \vdots & \ddots & \vdots \\
%        \eval{50-#1} & a_{\eval{50-#1}} & a_{\eval{50-(#1-1)}} & \cdots & a_{50} \\
%     \end{array} \right] \)}


\newcommand{\algoii}{
    \begin{algorithm}[H]
    \caption{Function \textit{generate}, that uses probabilistic grammars
        % ki uporablja verjetnostne % kontekstno-neodvisne gramatike}
        }
    % \label{algo:create}
    \raggedright
    \textbf{Input:} Probabilistic grammar $G=(N,T,R,S)$, 
    % nekon"cni 
        symbol $A \in N$ \\
    \textbf{Output:} Sentence $s$ in grammar $G$
    %   Seznam $ena"cbe$, ki vsebuje pare ena"cb in njihovih napak.
    \begin{algorithmic}[1]
    \Function{Generate}{$G$, $A$} %\Comment{Vsi vhodni parametri morajo biti opisani.}
    \State $s \gets [\,\,]$ 
    % \State Izberi naključno pravilo 
    \State Randomly choose rewriting rule 
    $(A \to A_1\ A_2 \cdots A_k ) \in R$
    % , kjer $\alpha = $
    \For{$i=1, ..., k$}
    \If{$A_i \in T$} 
    \State $s = s$.append($A_i$) 
    \Else 
    \State $s_i$ = \Call{Generate}{$G$, $A_i$}
    \State $s = s$.append($s_i$) 
    \EndIf
    \EndFor
    % \label{algo:pomembna-vrstica}
    \State \Return $s$
    \EndFunction
    \end{algorithmic}
    \end{algorithm}
}


\DeclareMathOperator{\Coverage}{Coverage} 

% % down 4.4.2017
\newcommand{\R}{\mathbb R}
% \newcommand{\N}{\mathbb N}
% \newcommand{\Z}{\mathbb Z}
\newcommand{\C}{\mathbb C}
\newcommand{\Q}{\mathbb Q}
\definecolor{softyellow}{rgb}{0.98,0.98,0.75}
\setbeamercolor{loweryel}{fg=black,bg=softyellow}
% % up 4.4.2017
\newcommand{\e}{\boldsymbol{e}}
\newcommand{\X}{\mathscr{X}}
\newcommand{\I}{\mathscr{I}}
\newcommand{\rhoo}{\boldsymbol{\rho}}
\newcommand{\q}{\boldsymbol{q}}
\newcommand{\1}{\boldsymbol{1}}
\newcommand{\0}{\boldsymbol{0}}
\newcommand{\N}{\mathbb{N}}
\newcommand{\Nom}{\mathbb{N}_0^m}
\newcommand{\Z}{\boldsymbol{Z}}
\newcommand{\Zn}{\boldsymbol{Z}_n}
\newcommand{\M}{\boldsymbol{M}}
\newcommand{\re}{\boldsymbol{r}}
% \newcommand{\r}{\boldsymbol{r}}
\renewcommand{\r}{\boldsymbol{r}}
\newcommand{\s}{\boldsymbol{s}}
% glej $$ \r $$ 
% \newcommand{\e}{\boldsymbol{e}}
\renewcommand{\e}{\boldsymbol{e}}

% \newtheorem{lemma}{Lemma}
% \newtheorem{definition}{Definition}
% \newtheorem{theorem}{Theorem}



%\setbeamercovered{invisible} %default
\setbeamercovered{transparent}
%\setbeamercovered{dynamic}


%\usepackage[dvipsnames]{color}

\usepackage[normalem]{ulem} % za strikeout (prečrtat besedo)
% na primer : \sout{Hello World}

% podatki
\title{Discovery of exact equations for integer sequences}
\author{Boštjan Gec}
\institute{mentor: prof. dr. Ljupčo Todorovski}

% tvoj izbran stil predstavitve
%\usetheme{Singapore}
\usetheme{Luebeck}
%\usecolortheme{crane}
%\usecolortheme{red}

\usepackage{color}
\setbeamercolor{structure}{fg=Bittersweet}
\setbeamercolor{structure}{fg=ForestGreen}
% \definecolor{sandyellow}{rgb}{0.98,0.98,0.75}
\definecolor{sandyellow}{HTML}{F4DF7A}
% \setbeamercolor{structure}{fg=sandyellow,use=black}
\setbeamercolor{structure}{fg=sandyellow}
\setbeamercolor{normal text}{fg=black}
\setbeamercolor{title}{fg=black}
\setbeamercolor{titlelike}{fg=black}
% \setbeamercolor{sidebar}{fg=black}
% \setbeamercolor{section in sidebar}{fg=black}
\setbeamercolor{palette primary}{fg=black}


% \usepackage{algorithmic}
\usepackage{algpseudocode}  % za psevdokodo
\usepackage{algorithm}
% \floatname{algorithm}{Algoritem}
% \renewcommand{\listalgorithmname}{Kazalo algoritmov}
\usepackage{multirow}

\usepackage{hyperref}

\begin{document}


\begin{frame}
  \maketitle
%  \fillblack{0.33}
\end{frame}


% \section{Equation discovery}
% \section{Verjetnostne gramatike}
% \section{Celo"stevilska zaporedja}

\begin{frame}{Introduction}
\begin{block}{Equation discovery for integer sequences with probabilistic grammars}
\begin{itemize}
	\item Equation discovery
	\item Probabilistic grammars
	\item Integer sequences
\end{itemize}
\end{block}
\end{frame}



% \begin{frame}{Experimental Evaluation (\textit{linrec})}
%  % \frametitle{Naslov prosojnice}
%  % \framesubtitle{Linrec}
%  % 	\begin{block}{Odkrivanje enačb za celoštevilska zaporedja z verjetnostnimi gramatikami}
     
%     \begin{algorithm}[H]
%         \caption{For experimental evaluation of SINDy for \textit{linrec} case
%         % ki uporablja verjetnostne % kontekstno-neodvisne gramatike}
%         }
%     % \label{algo:create}
%     \raggedright
%         \textbf{Input:} Matrix  $M = [X|\textbf{y}]$ of observations 
%         $X = [\textbf{a}_{n-1}|...|\textbf{a}_{n-19}]$ and target $\textbf{y} = \textbf{a}_n$
%     % nekon"cni 
%         % symbol $A \in N$ 
%         \\
%         \textbf{Output:} Equation $a_n = \sum\nolimits_{i=1}^{19}c_i a_{n-i}$ or Failed
%     %   Seznam $ena"cbe$, ki vsebuje pare ena"cb in njihovih napak.
%     \begin{algorithmic}[1]
%     \Function{SINDy's linrec}{M} %\Comment{Vsi vhodni parametri morajo biti opisani.}
%     % , kjer $\alpha = $
% % \State{output = Diofantos() }
% % \If {output is not Failed}
% %     \State{\Return output}
% % \EndIf
%     \For{$p=1, 2, ..., 19$}
%         \State{Add $col_p(X)$ to $M$ (i.e. $X = [X|M_p]$)}
%     % \State{output = Diofantos() }
%             \If {SINDy\_grid($M, d_{max}=1$) is not Failed}
%                 \State{\Return SINDy\_grid($M, d_{max}=1$)}
%         \EndIf
%     \EndFor
%     % \label{algo:pomembna-vrstica}
%     \State \Return Failed
%     \EndFunction
%     \end{algorithmic}
%     \end{algorithm}
%  \end{frame}




\begin{frame}{Experimental Evaluation (\textit{linrec})}
 % \frametitle{Naslov prosojnice}
 % \framesubtitle{Linrec}
 % 	\begin{block}{Odkrivanje enačb za celoštevilska zaporedja z verjetnostnimi gramatikami}
     
    \begin{algorithm}[H]
        \caption{For experimental evaluation of Diofantos for \textit{linrec} case
        }
    \raggedright
        \textbf{Input:} Matrix  $M = [X|\textbf{y}]$ of observations 
        $X = [\textbf{a}_{n-1}|...|\textbf{a}_{n-19}]$ and target $\textbf{y} = \textbf{a}_n$
        \\
        \textbf{Output:} Equation $a_n = \sum\nolimits_{i=1}^{19}c_i a_{n-i}$ or Failed
    \begin{algorithmic}[1]
    \Function{Diofantos's linrec}{M} %\Comment{Vsi vhodni parametri morajo biti opisani.}
    % , kjer $\alpha = $
% \State{output = Diofantos() }
% \If {output is not Failed}
%     \State{\Return output}
% \EndIf
    \For{$p=1, 2, ..., 19$}
        \State{Add $col_p(X)$ to $M$ (i.e. $X = [X|M_p]$)}
    % \State{output = Diofantos() }
            \If {Diofantos($M, d_{max}=1$) is not Failed}
                \State{\Return Diofantos($M, d_{max}=1$)}
        \EndIf
    \EndFor
    % \label{algo:pomembna-vrstica}
    \State \Return Failed
    \EndFunction
    \end{algorithmic}
    \end{algorithm}
 \end{frame}




\begin{frame}{Experimental Evaluation (\textit{linrec})}
 % \frametitle{Naslov prosojnice}
 % \framesubtitle{Linrec}
 % 	\begin{block}{Odkrivanje enačb za celoštevilska zaporedja z verjetnostnimi gramatikami}
     
    \begin{algorithm}[H]
        \caption{For experimental evaluation of SINDy for \textit{linrec} case
        % ki uporablja verjetnostne % kontekstno-neodvisne gramatike}
        }
    % \label{algo:create}
    \raggedright
        \textbf{Input:} Matrix  $M = [X|\textbf{y}]$ of observations 
        $X = [\textbf{a}_{n-1}|...|\textbf{a}_{n-19}]$ and target $\textbf{y} = \textbf{a}_n$
    % nekon"cni 
        % symbol $A \in N$ 
        \\
        \textbf{Output:} Equation $a_n = \sum\nolimits_{i=1}^{19}c_i a_{n-i}$ or Failed
    %   Seznam $ena"cbe$, ki vsebuje pare ena"cb in njihovih napak.
    \begin{algorithmic}[1]
    \Function{SINDy's linrec}{M} %\Comment{Vsi vhodni parametri morajo biti opisani.}
    % , kjer $\alpha = $
% \State{output = Diofantos() }
% \If {output is not Failed}
%     \State{\Return output}
% \EndIf
    \For{$p=1, 2, ..., 19$}
        \State{Add $col_p(X)$ to $M$ (i.e. $X = [X|M_p]$)}
    % \State{output = Diofantos() }
            \If {SINDy\_grid($M, d_{max}=1$) is not Failed}
                \State{\Return SINDy\_grid($M, d_{max}=1$)}
        \EndIf
    \EndFor
    % \label{algo:pomembna-vrstica}
    \State \Return Failed
    \EndFunction
    \end{algorithmic}
    \end{algorithm}
 \end{frame}



\begin{frame}{Experimental Evaluation (\textit{core})}
 % \frametitle{Naslov prosojnice}
 % \framesubtitle{Linrec}
 % 	\begin{block}{Odkrivanje enačb za celoštevilska zaporedja z verjetnostnimi gramatikami}
     
    \begin{algorithm}[H]
        \caption{For experimental evaluation of Diofantos for \textit{core} case
        }
    \raggedright
        \textbf{Input:} Matrix  $M = [X|\textbf{y}]$ of observations 
         $X = [\textbf{n}|\textbf{1}|\textbf{a}_{n-1}|...|\textbf{a}_{n-10}]$ 
         and target $\textbf{y} = \textbf{a}_n$
        \\
        \textbf{Output:} Polynomial equation $a_n = q(n, a_{n-1}, ... a_{n-10})$ or Failed
    \begin{algorithmic}[1]
    \Function{Diofantos's core}{M} %\Comment{Vsi vhodni parametri morajo biti opisani.}
    % , kjer $\alpha = $
% \State{output = Diofantos() }
% \If {output is not Failed}
%     \State{\Return output}
% \EndIf
        % \State{M = [\textbf{y}]}
    \For{$d_{max}=1, 2, 3$}
        \State{M = [\textbf{y}]}
        \For{$p=1, 2, ..., 10$}
            \State{Add $col_p(X)$ to $M$ (i.e. $X = [X|M_p]$)}
            \If {Diofantos([1], $M, d_{max}$) is not Failed}
                \State{\Return Diofantos([1], $M, d_{max}$)}
        \EndIf
        \EndFor
    \EndFor
    % \label{algo:pomembna-vrstica}
    \State \Return Failed
    \EndFunction
    \end{algorithmic}
    \end{algorithm}
 \end{frame}

\begin{frame}{Experimental Evaluation (\textit{core}) with SINDy}
 % \frametitle{Naslov prosojnice}
 % \framesubtitle{Linrec}
 % 	\begin{block}{Odkrivanje enačb za celoštevilska zaporedja z verjetnostnimi gramatikami}
     
    \begin{algorithm}[H]
        \caption{For experimental comparison of SINDy for \textit{core} case
        % ki uporablja verjetnostne % kontekstno-neodvisne gramatike}
        }
    % \label{algo:create}
    \raggedright
        \textbf{Input:} Matrix  $M = [X|\textbf{a}_n]$ of observations 
         $X = [\textbf{n}|\textbf{a}_{n-1}|...|\textbf{a}_{n-10}]$ 
         % and target $\textbf{y} = \textbf{a}_n$
    % nekon"cni 
        % symbol $A \in N$ 
        \\
        \textbf{Output:} Polynomial equation $a_n = q(n, a_{n-1}, ... a_{n-10})$ or Failed
    %   Seznam $ena"cbe$, ki vsebuje pare ena"cb in njihovih napak.
    \begin{algorithmic}[1]
    \Function{SINDy's core}{M} %\Comment{Vsi vhodni parametri morajo biti opisani.}
    % , kjer $\alpha = $
% \State{output = Diofantos() }
% \If {output is not Failed}
%     \State{\Return output}
% \EndIf
        \State{M = [\textbf{n}|\textbf{y}]}
    \For{$p=1, 2, ..., 10$}
        \State{Remove column \textbf{n} and add column $\textbf{a}_{n-p}$ to $M$}
        \If {SINDy\_grid($M, d_{max}$) is not Failed}
            \State{\Return SINDy\_grid($M, d_{max}=1$)}
        \EndIf
        \State{Add column of observations corresponding to $n$ to $M$}
        \If {SINDy\_grid($M, d_{max}$) is not Failed}
            \State{\Return SINDy\_grid($M, d_{max}=1$)}
        \EndIf
    \EndFor
    % \label{algo:pomembna-vrstica}
    \State \Return Failed
    \EndFunction
    \end{algorithmic}
    \end{algorithm}
 \end{frame}

\end{document}

% % \begin{frame}{Algoritem za odkrivanje ena"cb}
% % \begin{itemize}
% % [<+->]
% %     \item dsa
% % \end{itemize}
% % \end{frame}

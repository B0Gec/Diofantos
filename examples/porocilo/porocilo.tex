\documentclass[10pt,a4paper]{article}
\usepackage[utf8]{inputenc}
% \usepackage[T1]{fontenc}
\usepackage[slovene]{babel}

\usepackage{multirow}

\usepackage{amsmath}
% \usepackage{amsfonts}
% \usepackage{amssymb}

% \usepackage{graphics}
\usepackage{graphicx}

% \usepackage{caption}
% \captionsetup[table]{name=Tabela}



\title{Poro"cilo okrivanja 1-D ode v Lorenzovemu sistemu}


\begin{document}
	\maketitle
    
Odkrivanje 1-D ena"cb sem poganjal na Lorenzovemu sistemu enacb:
$$
\frac{dx}{dt} = \sigma (y-x),
$$ $$
\frac{dy}{dt} = x(\rho-z) - y,
$$ $$
\frac{dz}{dt} = xy - \beta z.
$$

Pri za"cetnih pogojih $x_0:=0.1, y_0:=0.3, z_0:=0.4$.
Najprej sem pognaj pri naklju"cno parametrih:
$\sigma:=1.3, \rho:=-15, \beta:=3.4$

Za"cetni pogoji so isti v vseh primerih skozi celotno poro"cilo.
Izbrani parametri se splo"sno smatrajo kot nenormalni, saj je eden izmed parametrov
($\rho$) negativen.
Podatkovno mno"zico sem generiral tako, da sem simuliral Lorenzov sistemu:

prva enacba  $dx/dt = \sigma*(y-x)$: najde re"sitev 
$ -1.3*x + 1.3*y$ ali 
v 50 samplih, resitev ima napako reda $10**(-9)$.


druga enacba  $dx/dt = \sigma*(y-x)$: najde re"sitev v 4500 ali 6500 samplih, resitev 
$10*x*z -10*x +2*y +0.5$ oz. $10*x*z -10*x +2*y$ ima napako
reda $10**(-6)$ oz. $10**(-4)$. 

Tako velik odmik od pravilne resitve  
$-15*x-x*z-y$ pripisujem trenutno nastavljeni omejitvi v implementaciji optimizacijskega
algoritma, ki omejuje parametre na interval [-10, 10]. Parameter v "clenu $-10*x$ je tako
lahko po absolutni vrednosti najve"c 10, torej ne more biti -15, kot je v izvorni ena"cbi.
Predvidevam, da se zato zgodi kompenzacija nad ostalimi parametri v ostalih "clenih ena"cbe.
Predvidevam "se, da se bo pri rahljanju omejitve iz [-10, 10] na [-20, 20] napaka popravila
na napako reda $10**(-9)$ kot pri ostalih dveh ena"cbah.

tretja ena"cba  $dx/dt = \sigma*(y-x)$: najde re"sitev 
$ -1.3*x + 1.3*y$ ali 
v 50 samplih, resitev ima napako
reda $10**(-9)$.

Sledi poro"cilo o poganjanju pri kaoti"cnih parametrih:
$\sigma:=10, \rho:=28, \beta:=8/3$

Ker je $\rho$ po absolutni vrednosti spet ve"cji od 10, tj. od nastavljenih 
mej za parametre optimizacijskega algoritma in so ostali dve vrednosti 
znotraj mej, napovedujem, da bodo v najbolj"sem primeru podobni rezultati kot 
v nekaoti"cnem primeru. Oziroma, pri drugi ena"bi pri"cakujem 
$\sigma* = 10, \rho*= 10, \beta*=7$. 


\subsection{Kaoti"cni rezultati}

V odkrivanju prve ena"cbe algoritem odkrije ena"cbo
$-9.85764357227234*x + 9.9333747564978*y$ oz. 
$-9.55829580188787*x + 9.78920618974904*y + 0.0232261285460231$
v manj kot 50 vzorcih, re"sitev ima napako velikosti
$2.94709382690573e-06$ oz. $2.4772067855792343e-06$.

V odkrivanju druge ena"cbe algoritem odkrije ena"cbo
$-9.85764357227234*x + 9.9333747564978*y$ oz. 
$-9.55829580188787*x + 9.78920618974904*y + 0.0232261285460231$
v 6500 samplih, re"sitev ima napako velikosti
$2.94709382690573e-06$.
% -0.877950876789105*x*z + 9.99535400929563*x + 7.38550161602604*y - 0.617064542958862; error: 0.00018803715641311185


model: 0.991337569095305*x*y - 2.56521319047224*z                            ; error: 9.178140365155879e-05
\section{Poro"cilo v tabeli}
\newcommand{\asui}{rf-100,10  }
 \newcommand{\asuii}{glm        }
 \newcommand{\asuiii}{glm        }
 \newcommand{\asuiv}{squared+glm}
 \newcommand{\nui}{625}
 \newcommand{\nuii}{562}
 \newcommand{\nuiii}{500}
 \newcommand{\nuiv}{438}
 \newcommand{\eui}{0.00000}
 \newcommand{\euii}{0.00000}
 \newcommand{\euiii}{0.00000}
 \newcommand{\euiv}{0.00000}
 \newcommand{\nti}{625}
 \newcommand{\ntii}{563}
 \newcommand{\ntiii}{500}
 \newcommand{\ntiv}{438}
 \newcommand{\eti}{0.00000}
 \newcommand{\etii}{0.00000}
 \newcommand{\etiii}{0.00000}
 \newcommand{\etiv}{0.00000}

% \begin{table}[h]
% \begin{tabular}{ *{5}{c|}c  }		
%   &  Algoritem  &
%     \multicolumn{2}{c|}{U"cna mno"zica}  & \multicolumn{2}{c}{Testna mno"zica} \\
%      Segment & strojnega u"cenja & "Stevilo primerov & Napaka & "Stevilo primerov & Napaka \\
%      \hline
%      \#1 & \asui   & \nui   & \eui   & \nti   & \eti  \\
%      \#2 & \asuii  & \nuii  & \euii  & \ntii  & \etii  \\
%      \#3 & \asuiii & \nuiii & \euiii & \ntiii & \etiii  \\
%      \#4 & \asuiv  & \nuiv  & \euiv  & \ntiv  & \etiv  \\
% \end{tabular}
% \caption{Tabela s poro"cilom o rezultatih}
% \end{table}

\end{document}